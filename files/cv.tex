%%%%%%%%%%%%%%%%%%%%%%%%%%%%%%%%%%%%%%%%%
% Medium Length Professional CV
% LaTeX Template
% Version 2.0 (8/5/13)
%
% This template has been downloaded from:
% http://www.LaTeXTemplates.com
%
% Original author:
% Trey Hunner (http://www.treyhunner.com/)
%
% Important note:
% This template requires the resume.cls file to be in the same directory as the
% .tex file. The resume.cls file provides the resume style used for structuring the
% document.
%
%%%%%%%%%%%%%%%%%%%%%%%%%%%%%%%%%%%%%%%%%

%----------------------------------------------------------------------------------------
%	PACKAGES AND OTHER DOCUMENT CONFIGURATIONS 
%----------------------------------------------------------------------------------------
 
\documentclass{resume} % Use the custom resume.cls style 
\usepackage{hyperref}
\usepackage[left=0.4 in,top=0.2 in,right=0.4 in,bottom=0.3in]{geometry} % Document margins
\newcommand{\tab}[1]{\hspace{.2667\textwidth}\rlap{#1}}
\newcommand{\itab}[1]{\hspace{0em}\rlap{#1}}
\name{Stuart Burrell} % Your name 
\address{Carnegie Scholar \\ University of St Andrews \\ United Kingdom} % Your address 
%\address{123 Pleasant Lane \\ City, State 12345} % Your secondary addess (optional) 
\address{\href{https://www.linkedin.com/in/stuartburrell}{LinkedIn} \\ \href{https://github.com/stuartburrell}{GitHub} \\
\href{https://stuartburrell.github.io/}{Website}}
% Your phone number and email
\begin{document}  

%----------------------------------------------------------------------------------------
%	EDUCATION SECTION
%----------------------------------------------------------------------------------------
\vspace{.1in}
\begin{tabular}{ @{} >{\bfseries}l @{\hspace{6ex}} l }  
Email & stuartburrell1994@gmail.com \\
Phone & +44 7979 004965 \\
Languages & English, Basic German, Basic Japanese \\
Programming & Python, R, Git/GitHub, GAP, Unix, LaTeX, Maple.
\end{tabular}   
\vspace{.1in}

\begin{rSection}{Education}


{\bf PhD, Mathematics}, 
\emph{University of St Andrews} \hfill {October 2017 - Present}
\\ 
Fractal geometry, dynamical systems and dimension theory. \\
{\textbf{Scottish Mathematical Sciences Training Center}, \\
{Graduate courses in Algebra, Analysis, Topology, Probability and Statistics}} \hfill 2016 - 2018  \\
{\textbf{Academy for PhD Training in Statistics}, \\
{Graduate courses in Statistics at the Universities of Cambridge, Durham and Glasgow.}} \hfill 2016 - 2017 \\ 
{\bf Machine Learning Engineer Nanodegree}, \emph{Udacity} \hfill {2016 - 2017}

{\bf MMath, Mathematics}, First Class Honours (93\%/GPA 4.0),
\emph{University of St Andrews} \hfill {2012 - 2016}

\end{rSection} 

\begin{rSection}{Publications}
\textbf{The dimensions of inhomogeneous self-affine sets (with J.M. Fraser)}, \href{https://arxiv.org/abs/1807.08694}{arXiv} \hfill July 2018\\ \emph{submitted.}\\ 
\textbf{On the dimension and measure of inhomogeneous attractors}, \href{https://arxiv.org/abs/1805.00887}{arXiv} \hfill May 2018\\ 
\emph{Real Analysis Exchange} (to appear).
\end{rSection}

\begin{rSection}{Software Contributions}
\href{https://www.gap-system.org/Packages/semigroups.html}{\textbf{GAP Semigroups package}}\hfill 2016\\ Methods to decide the order and torsion problems of natural and tropical matrix semigroups.\\ 
\href{https://www.gap-system.org/Packages/digraphs.html}{\textbf{GAP Digraphs package}}\hfill 2016\\ Methods for computing the simple circuits of a digraph.
\end{rSection}

\begin{rSection}{EXPERIENCE}
%\textbf{Mathematics Tutor}\\
%University of St Andrews \hfill September 2016 - Present \\
\textbf{Research in Statistical Ecology and Machine Learning} \hfill 2016-2017\\
\emph{Centre for Research into Ecological and Environmental Modelling}, {St Andrews.}\\
\textbf{GAP Software Development}\hfill Summer 2016\\
\emph{School of Mathematics and Statistics, University of St Andrews}, St Andrews.\\
\textbf{Laidlaw Internship in Research and Leadership} \hfill Summer 2015\\
\emph{School of Mathematics and Statistics, University of St Andrews}, St Andrews.\\
\textbf{Undergraduate Research Summer School} \hfill Summer 2014\\
\emph{School of Mathematics and Statistics, University of St Andrews}, St Andrews.\\

%\textbf{Research Assistant in Pure Mathematics} \\
%School of Mathematics and Statistics, University of St Andrews\\
%\emph{Average distances in generalized Cantor sets}.\hfill Summer 2014
% and inescapable cores of synchronising transducers.
\end{rSection}

%----------------------------------------------------------------------------------------
%	TECHNICAL STRENGTHS SECTION
%----------------------------------------------------------------------------------------



%-------------------------------------------------------------------------------
%	PROJECTS

\begin{rSection}{Awards and Scholarships} \itemsep -3pt     
\textbf{Poster Award}, \emph{Fractals and Stochastics 6} \hfill October 2018 \\   
\textbf{Carnegie PhD Scholarship} \hfill 2016-2020 \\
\textbf{Duncan Prize}, \emph{University of St Andrews} \hfill {2016} \\
\textbf{Sanderson Prize}, \emph{University of St Andrews} \hfill {2016} \\
\textbf{The Arthur Hinton Read Memorial Prize}, \emph{University of St Andrews} \hfill {2016}\\
\textbf{ODSC Machine Learning/Deep Learning Scholarship} \hfill {2016}\\
\textbf{IMA Graduate Prize Winner}, \emph{Institute of Mathematics}\hfill {2016}\\
\textbf{Graduate Medal in Mathematics}, \emph{University of St Andrews}\hfill {2016}\\
\textbf{The Principal's Scholarship for Academic Excellence}, \emph{University of St Andrews} \hfill {2012-16} \\
\textbf{The University Scholarship for Research and Leadership}, \emph{University of St Andrews} \hfill {2015}\\
\textbf{The Deans' List for Academic Excellence}, \emph{University of St Andrews} \hfill {2012-16}\\
\textbf{Gold Duke of Edinburgh Award} \hfill {2013}
% \textbf{The Head Teachers Award} \emph{an award given to a single student upon graduating high school.}
% \textbf{Japanese Speech Competition}
\end{rSection}


%	INTERNSHIP/TRAININGS 
%----------------------------------------------------------------------------------------
\begin{rSection}{Talks}
\textbf{Foundations and applications of fractal geometry}\hfill January 2019 \\ \emph{Postgraduate Interdisciplinary Mathematical Symposium}, Glen Esk  \\
\textbf{Dimension drop and inhomogeneous self-affine sets} \hfill October 2018 \\ \emph{Pure Analysis Seminar}, St Andrews.\\
\textbf{The dimensions of inhomogeneous self-affine sets (poster presentation)} \hfill October 2018 \\ \emph{Fractals and Stochastics VI}, Bad Herrenalb, Germany.\\
\textbf{A brief note on the dimension of inhomogeneous attractors} \hfill September 2018 \\\emph{Dynamic Days Europe}, Loughborough.\\
\textbf{A universal upper bound on the dimension of inhomogeneous attractors} \hfill June 2018 \\\emph{British Mathematical Colloquium}, St Andrews.\\
\textbf{How big are inhomogeneous attractors?} \hfill May 2018 \\\emph{Edinburgh Mathematical Society PG Student Meeting}, The Burn, Glen Esk\\
\textbf{On the dimension and measure of inhomogeneous attractors} \hfill April 2018 \\\emph{Pure Analysis Seminar}, St Andrews.\\
\textbf{An introduction to inhomogeneous attractors}  \hfill April 2018\\ \emph{Pure Postgraduate Seminar}, St Andrews.\\
\textbf{An introduction to iterated function systems}\hfill February 2018 \\ \emph{Postgraduate Interdisciplinary Mathematical Symposium}, Glen Esk  \\
\textbf{Dimension of inhomogeneous self-conformal and self-affine sets}\hfill January 2018\\ \emph{School Research Day}, St Andrews.  \\
\textbf{Inhomogeneous attractors and upper box dimension}\hfill December 2017\\ \emph{Insitut Mittag-Leffer}, Stockholm, Sweden. \\
\textbf{Trace contrast methods in acoustic space}\hfill January 2017\\ \emph{African Institute of Mathematical Sciences}, Cape Town, South Africa.  \\
\textbf{An introduction to trace contrast methods}\hfill January 2017 \\ \emph{University of Cape Town}, Cape Town, South Africa.\\
\textbf{Sequential Monte Carlo in population dynamics}\hfill April 2016\\ \emph{School of Mathematics and Statistics}, St Andrews. \\
\textbf{Artificial intelligence and G\"odel's incompleteness theorems}\hfill February 2015 \\ \emph{School of Mathematics and Statistics}, St Andrews. \\
%\textbf{Japanese speech competition finalist}, \emph{Japanese Embassy, London} \hfill 2010
\end{rSection}

\begin{rSection}{Outreach}

\textbf{Sundials, dimensions and fractal projections}\hfill February 2019 \\ \emph{The Royal College of Physicians and Surgeons}, Glasgow  \\
\textbf{Positive problem solving} \hfill February 2019 \\
\emph{First Chances Scheme}, University of St Andrews \\
\textbf{The journey to a PhD in Mathematics} \hfill July 2018 \\\emph{UWS Summer STEM Academy}, Glasgow.\\
\textbf{Fractal geometry in nature and art} \hfill February 2018\\ \emph{The Royal Society of Edinburgh}, Edinburgh.\\
\end{rSection}

\begin{rSection}{Conferences} \itemsep -3pt 
{\textbf{Fractal Geometry and Stochastics VI}} \hfill October 2018\\ \emph{Evangelische Akademie Baden}, Bad Herrenalb, Germany\\ 
{\textbf{CMI at 20}} \hfill September 2018\\ \emph{Clay Mathematics Institute}, Oxford.\\ 
{\textbf{Dynamic Days Europe}} \hfill September 2018\\ \emph{University of Loughborough}, Loughborough.\\ 
{\textbf{Thermodynamic Formalism in Dynamical Systems}} \hfill June 2018\\ \emph{International Centre for Mathematical Sciences}, Edinburgh.\\ 
{\textbf{British Mathematical Colloquium}} \hfill June 2018\\ \emph{University of St Andrews}, {St Andrews.} \\ 
{\textbf{Fractals and Dimensions}} \hfill December 2017\\ \emph{Insitut Mittag-Leffler},{ Stockholm, Sweden} \\ 
{\textbf{ODSC (Machine Learning/Deep Learning)}} \hfill October 2016\\ London.  \\ 
\end{rSection}  
\begin{rSection}{Teaching}
All available student feedback data is included and reported on a scale of 1 (excellent) to 5 (poor) in the cateogires of Explanation (E), Organisation (O) and Availability (A).\\

\textbf{Mathematics and Statistics Tutor}, \href{https://www.st-andrews.ac.uk/capod}{\emph{CAPOD, University of St Andrews}}, \hfill 2016 - Present\\
\textbf{MT2502 Analysis}, \emph{Tutor, University of St Andrews} \hfill Autumn 2018 \\
E = 1.09, O = 1.33, A = 1.0\\
\textbf{MT2000 Python}, \emph{Demonstrator, University of St Andrews} \hfill Autumn 2018 \\
\textbf{Math Base}, \emph{Tutor, University of St Andrews} \hfill Autumn 2018 \\
\textbf{MT1002 Mathematics}, \emph{Demonstrator, University of St Andrews} \hfill Spring 2018 \\
\textbf{MT1002 Mathematics}, \emph{Tutor/Demonstrator, University of St Andrews} \hfill Autumn 2017 \\
E = 1.62, O = 1.65, A = 1.17\\
\textbf{MT1003 Pure and Applied Mathematics}, \emph{Tutor (temporary), University of St Andrews} \hfill Spring 2017 \\
\textbf{MT2508 Statistical  Inference}, \emph{Demonstrator, University of St Andrews} \hfill Spring 2017 \\
\textbf{MT2504 Combinatorics and Probability}, \emph{Tutor/Demonstrator, University of St Andrews} \hfill Autumn 2016 \\
E = 1.17, O = 1.15, A = 1.17\\
\textbf{MT2000 Introduction to Python}, \emph{Demonstrator, University of St Andrews} \hfill Autumn 2016 \\
\end{rSection}

\begin{rSection}{Workshops} \itemsep -3pt  
{\textbf{Postgraduate Interdiscipliniary Writing Workshop}}\hfill August 2018\\ \emph{University of St Andrews}, The Burn, Glen Esk.  \\ 
{\textbf{Summer School in Dynamics (Introductory and Advanced)}}\hfill July 2018\\ \emph{International Centre for Theoretical Physics}, Trieste, Italy.  \\ 
{\textbf{Postgraduate Interdisciplinary Mathematical Symposium}}\hfill January 2018\\ \emph{University of St Andrews}, The Burn, Glen Esk.  \\ 
{\textbf{Statistical Methods in Gibbon Conservation}} \hfill August 2017\\ \emph{Centre for Research into Ecological and Environmental Modelling}, {St Andrews.} \\ 
{\textbf{Estimating Animal Abundance and Density using Acoustic Data}}\hfill January 2017\\ \emph{University of Cape Town}, Cape Town, South Africa.  \\ 
{\textbf{Scottish Mathematical Craining Center Symposium}\hfill October 2016\\  {Perth.}} \\
{\textbf{CoDiMa software carpentry}}\hfill October 2016 \\ \emph{International Centre for Mathematical Sciences}, Edinburgh. \\
{\textbf{Advanced PAMGuard}}\hfill October 2016\\ \emph{PAMTech}, {Edinburgh.}  \\
{\textbf{Spatial Capture-Recapture Methods} \hfill August 2016\\ \emph{Centre for Research into Ecological and Environmental Modelling}, {St Andrews.}} \\
\textbf{Laidlaw Leadership Program}\hfill 2015\\ \emph{endorsed by the Institute of Management}, St Andrews.
\end{rSection}  


\begin{rSection}{Professional Responsibilities}
\textbf{Carnegie Trust Scholars Network}, \emph{Event organiser} \hfill 2019 - Present\\
\textbf{Postgraduate Interdisciplinary Mathematical Symposium}, \emph{Organiser} \hfill 2018 - 2019\\
\textbf{Scottish Mathematical Sciences Training Centre Council}, \emph{Student Representative} \hfill 2016 - 2017\\
\end{rSection}



%\begin{rSection}{Professional Responsibilities and Volunteering}
%\textbf{SMSTC Council Student Representative} \hfill 2016 - 2017\\
%\textbf{Secretary and Team Captain}, \emph{University Chess Society} %\hfill 2014 - 2015\\
%\textbf{Committee Member}, \emph{University Hiking Society} \hfill %2014 - 2015\\
%\textbf{Fundraiser}, \emph{Childreach International: Expedition %Everest}
%\hfill 2012 - 2013
%\end{rSection}

%\begin{rSection}{Professional Development}
%\textbf{ID5XXX Teaching}
%\textbf{ID6XXX Teaching}
%\end{rSection}


%%%%%% optional %%%%%%%%%%%%%%%%%%%%%%%%%%%%%%%%%%%%%%%%%%%%%%%%

%\begin{rSection}{EXPERIENCE}
%\textbf{Mathematics Tutor}\\
%University of St Andrews \hfill September 2016 - Present \\
%\textbf{GAP Software Development}\hfill 2015 - 2016\\
%\emph{School of Mathematics and Statistics, University of St Andrews}\\
%- Comprehensive literature review on Tropical Matrix Semigroups, and %the Spectral Theory of tropical matrices.\\
%- Implemented a series of algorithms for the GAP Semigroups package %3.0 release that decide the cardinality of certain tropical matrix %semigroups.\\
%- Implemented Johnson's Algorithm et al. for the GAP Digraphs package, %version 0.6.0.\\
%\textbf{Doctoral Researcher in Statistics and Machine Learning} \hfill %2016-2017\\
%\emph{Centre for Research into Ecological and Environmental Modelling, %St Andrews}\\
%- Implemented speaker recognition methods for avian and cricket %acoustic data.\\
%- Developed distance sampling methodology for use in triangulation %surveys.\\
%- Processed and analysyed acoustic data to derive abundance from %temporal clustering.\\
%\textbf{Laidlaw Internship in Research and Leadership} \hfill Summer %2015\\
%\emph{School of Mathematics and Statistics, University of St Andrews}\\
%- Researched the torsion problem for the group generated by the class %of synchronising transducers.\\
%- Attended a leadership program endorsed by the Institute of %Management.
%\textbf{Research Assistant in Pure Mathematics} \\
%School of Mathematics and Statistics, University of St Andrews\\
%\emph{Average distances in generalized Cantor sets}.\hfill Summer 2014
% and inescapable cores of synchronising transducers.
%\end{rSection}

%\begin{rSection}{PROJECTS}
%\textbf{Thesis: The Order Problem for Tropical Matrix Semigroups}\\
%\\
%\textbf{Sequential Monte Carlo in Population Dynamics},
%\emph{MMath Thesis, University of St Andrews}\\
%- Awarded Duncan Prize for best thesis in the department.\\
%- Created Bootstrap and Auxiliary particle filters for density %dependent population models.\\
%- Developed theory and testable hypotheses for enhancements including a diagnostic method to achieve real-time optimization through variation of mutation rates in particle swarm.\\
%\textbf{Creating Customer Segements from Spending Data},\emph{Unsupervised Learning Project, Udacity.}\\
%- Implemented PCA to find key signals to indicate customer spending.\\
%- Identified two distinct groups of customers allowing for more effective A/B testing.
%\textbf{Predicting Boston House Prices}, \emph{Supervised Learning, Udacity}\\
%\textbf{Designing a Student Intervention System}, \emph{Supervised Learning, Udacity.}
%\\
%\textbf{Training a Smart Cab to Drive}, \emph{Reinforcement Learning %Project, Udacity}\\
%\textbf{}
%\end{rSection}

\end{document}
